\documentclass[10pt,aspectratio=169]{beamer}

\mode<presentation>
{
    \usetheme[sectionpage, subsectionpage]{RTG2583}
}

% Standard packages
\usepackage[T1]{fontenc}
\usepackage[utf8]{inputenc}
\usepackage{lmodern}
% \usepackage{microtype}
\usepackage{amsmath, amsfonts, amssymb, amsthm}
\usepackage[justification=centering]{caption} %
\captionsetup[figure]{labelformat=empty}% redefines the caption setup of the figures environment in the beamer class.
\captionsetup{belowskip=0pt}

\usepackage{hyperref}
\definecolor{my_blue}{RGB}{0,80,200} % 0,77,128
\hypersetup{%
    citecolor     = my_blue,% Color of citations % 
    urlcolor      = my_blue,%  Color of external urls, i used my_lightblue before
}
\usepackage{enumerate}
\usepackage[export]{adjustbox}
\usepackage{multimedia}
\usepackage{dsfont} % for indicator function 
\usepackage{caption}
\usepackage{subcaption}
\usepackage{biblatex}
\addbibresource{literature.bib}
% Custom commands
\usepackage{algcompatible} % for long algorithms, algorithmicx
\usepackage{algorithm} % for algorithms
% \usepackage[symbol]{footmisc}
\usepackage{array} % draw matrices
\newcommand*{\vertbar}{\rule[-.5ex]{0.5pt}{5ex}}
\newcommand*{\horzbar}{\rule[.5ex]{5ex}{0.5pt}}
\usepackage{booktabs} % for fancy tables
\newcommand{\deriv}{\mathrm{d}} % new command for derivative d
\newcommand{\vect}[1]{\boldsymbol{#1}} % for bolding x 
\usepackage{tikz} % to create pictures 
\usetikzlibrary{positioning,shapes,arrows.meta}
\tikzset{
    process1/.style={rectangle, rounded corners, minimum width=4cm, minimum height=2cm, text width=8cm,text centered, draw=black, fill=red!30},
    process2/.style={rectangle, rounded corners, minimum width=4cm, minimum height=2cm, text width=5cm,text centered, draw=black, fill=orange!30},
    process3/.style={rectangle, rounded corners, minimum width=4cm, minimum height=2cm, text width=5cm,text centered, draw=black, fill=green!30},
    arr/.style={thick,-stealth}
    } % for the flow chart
%Title setup
%Title setup
\title[PhD project]{Technical Talk - PhD Project}
\subtitle{}
\author[Kamal Sharma]{Kamal Sharma}
\institute[UHH]{Department of Mathematics, University of Hamburg}

\date{February 13, 2026}


% Start document
\begin{document}
\begin{frame}[noframenumbering]
    \maketitle    
\end{frame}
\section{Introduction}
\begin{frame}{What is it about?}
\onslide<1->{\large \textcolor{teal}{Development} of a \textcolor{purple}{structure-preserving} \textcolor{orange}{idealized} \textcolor{orange}{stochastic climate model}}
\vspace{5mm}
\begin{itemize}
    \item<2-> \textcolor{orange}{Climate model:} set of equations modeling climatic events/processes
    \begin{itemize}
        \item<3-> Our model: simulates ocean-atmosphere interactions
    \end{itemize}
    \vspace{0.1cm}
    \item<4-> \textcolor{orange}{Idealized:} we make approximations!
    \begin{itemize}
        \item<5-> 2D model: solves for velocity, temperature, and pressure
    \end{itemize}
    \vspace{0.1cm}
    \item<6-> \textcolor{orange}{Stochastic:} has stochastic/random terms (think of Brownian motion)
    \vspace{0.1cm}
    \item<7-> \textcolor{purple}{Structure-preserving:} preserves underlying geometric and physical structure
    \vspace{0.1cm}
    \item<8-> \textcolor{teal}{Development:} modeling, analysis, and \textcolor{teal}{simulation}
\end{itemize} 
\vspace{5mm}
\onslide<9>{\large \textbf{We are solving stochastic PDEs which model ocean-atmosphere interactions!}}
\end{frame}
\end{document}

